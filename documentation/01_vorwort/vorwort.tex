\chapter{Kurzbeschreibung}
Diese Projektdokumentation behandelt das Konzept und die Implementierung der plattformunabhängigen STH-App für den Austausch zwischen Spielern und Sportmanagern.
Dabei werden zunächst die wichtigsten Voraussetzungen wie die Teambildung, die Projektinitiierung und die Projektskizze beschrieben.
Das Projektteam besteht aus vier Projektmitgliedern: Jonas Waigel, Fitim Makolli, Hasan Deveci und Vatsegkan Zournatsidis.
Zusätzlich hat das Team Jonas Waigel als Projektleiter gewählt.
Durch ihre einzigartigen Fähigkeiten, Erfahrungen und die gemeinsame Vision, eine innovative Smartphone-App zu gestalten, sowie ihre Leidenschaft für Sport, erlangte das Team den notwendigen Spirit und die Dynamik, um ein herausragendes Produkt zu erschaffen.
\newline
Die STH-App, kurz für „Sports Talent Hub“, wurde entwickelt, um Sportlern und Sportmanagern eine reibungslose Interaktion und Vernetzung zu bieten.
Spieler haben die Möglichkeit, ihre Profile zu erstellen, ihre sportlichen Erfolge und Höhepunkte zu teilen und mit möglichen Managern in Verbindung zu treten.
Sportmanager können gleichzeitig Talente identifizieren, kontaktieren und für unterschiedliche sportliche Projekte einstellen.
\newline
Nach vielen Recherchen und Überlegungen entwickelte das Projektteam eine Smartphone-App, die plattformunabhängig ist und den Austausch und das Anwerben zwischen Spielern und Sportmanagern ähnlich der Social Media App Instagram ermöglicht.
Die STH-App zielt darauf ab, eine Plattform zu schaffen, die effizient und benutzerfreundlich ist und Sportlern sowie Managern die Kommunikation und Zusammenarbeit in einem dynamischen und interaktiven Umfeld ermöglicht.
\newline
Das Konzept und die Umsetzung eines MVP (Minimum Viable Product) für die Anwendung sollen zwischen dem 10. März 2024 und dem 5. Juli 2024 entwickelt und in dieser Dokumentation erläutert werden.
Damit das Projekt flexibel auf Veränderungen reagieren und kontinuierliche Verbesserungen umsetzen kann, wird es agil in einer angepassten Form von Scrum durchgeführt.
Die Gründung eines Teams und eine deutliche Verteilung der Rollen innerhalb des Projekts sind für den Erfolg von großer Bedeutung.
Jeder Teilnehmer des Projekts bringt bestimmte Kompetenzen und Erfahrungen mit sich, die das gesamte Projektteam stärken.
Als Projektleiter übernimmt Jonas Waigel zusätzlich zur App-Entwicklung die Koordination der Tätigkeiten und stellt sicher, dass die Zeitpläne und Ziele eingehalten werden.
Die erfolgreiche Implementierung der App wird wesentlich von Fitim Makolli, Hasan Deveci und Vatsegkan Zournatsidis durch ihre Fachkenntnisse in den Feldern Entwicklung, Design und Qualitätssicherung beeinflusst.