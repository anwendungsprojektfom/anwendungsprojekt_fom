\chapter{Vorwort}
Diese Projektdokumentation behandelt das Konzept und die Implementierung der plattformunabhängigen SportTalentHub App (kurz STH) für den Austausch zwischen Spieler und Sportmanager.
Dabei werden zunächst die wichtigsten Voraussetzungen wie die Teambildung, die Projektinitiierung und die Projektskizze beschrieben.
Das Projektteam besteht aus vier Projektmitgliedern (Jonas Waigel, Fitim Makolli, Hasan Deveci und Vatsegkan Zournatsidis). 
Zusätzlich hat das Team Jonas Waigel als Projektleiter gewählt.
Durch die einzigartigen Fähigkeiten, Erfahrungen, die gemeinsame Vision, eine Smartphone-App zu gestalten und die Leidenschaft für Sport gab dem Team den notwendigen Spirit und die Dynamik, um ein tolles Produkt zu erschaffen.
\newline
Viele Recherchen, Überlegungen und die Begeisterung für Sport hat das Projektteam auf die Idee gebracht, eine plattformunabhängige Smartphone-App zu implementieren, die ähnlich zur Social Media App Instagram den Austausch und das Anwerben zwischen Spielern und Sportmanagern ermöglicht.
\newline
Dabei war es dem Team stets wichtig, dass zunächst anstehende Aufgaben detailliert besprochen werden und ein gemeinsames Einverständnis über Aufgaben und Themen herrschte.
Das Konzept und die Implementierung eines MVP der App soll im Zeitrahmen von 10.03.2024 bis 06.07.2024 entstehen und in dieser Dokumentation beschrieben werden.
Das Projekt soll agil in einer auf das Projekt abgewandelten Form von Scrum durchgeführt werden.


