\chapter{Implementierung des Chat-Screens}
Zu jeder modernen Community Applikation gehört auch die Chatfunktionalität immer dazu. Der Austausch zwischen zwei oder mehreren Menschen ist besonders in der STH App wichtig, um die Kommunikation zwischen Spielern und Managern zu ermöglichen. Die Anforderungen an den Chat-Screen standen schnell fest. Die Funktionalität muss mindestens den gleichen Interaktionsumfang ermöglichen, wie vergleichbare Apps wie z.B. Instagram, Snapchat oder X (ehemalig Twitter).

\section{Anforderungen an den Chat-Screen}
Die spezifischen Anforderungen für die Chat-Funktion in der STH App wurden im Projektteam gemeinsam ausgearbeitet. Dabei wurden zunächst vergleichbare Applikationen genauer beleuchtet und die Mindestanforderungen anhand der dort gesichteten Funktionen definiert. Das Chatten in der STH App muss mittels verschlüsselte eins zu eins Verbindung funktionieren, in der zwei Menschen miteinander in einem Chat-Raum kommunizieren können. Zudem muss es auch die Möglichkeit geben Gruppenchats zu erstellen und mehrere Menschen dazu einzuladen. Auch hierbei müssen die Verbindungen stets verschlüsselt sein. Innerhalb eines Chats muss jeder Nutzer Nachrichten senden und empfangen können. Zu den Nachrichten gehören Textnachrichten, GIFs, Bilder und Videos. Jeder Nutzer muss gesendete und empfangene Nachrichten im Chat sehen können und die Chats müssen mit ihrem kompletten Verlauf beim ein- und ausloggen erhalten und neu geladen werden können. 