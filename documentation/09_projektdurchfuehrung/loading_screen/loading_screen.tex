\chapter{Ladebildschirm}

Der Ladebildschirm ist ein essenzieller Bestandteil der Benutzererfahrung unserer STH-App (SportTalentHub). Er erscheint, bevor der Nutzer die Hauptfunktionen der App nutzen kann, und dient als Zwischenbildschirm, um dem Nutzer zu signalisieren, dass die App lädt.

\section{Ziele des Ladebildschirms}

Der Ladebildschirm hat mehrere wichtige Funktionen:

\begin{itemize}
    \item \textbf{Visuelles Feedback:} Er zeigt dem Nutzer, dass die App aktiv ist und lädt, wodurch eine bessere Benutzererfahrung gewährleistet wird.
    \item \textbf{Zeitliche Steuerung:} Der Ladebildschirm wurde so gestaltet, dass er genau vier Sekunden lang angezeigt wird. Dies gibt der App ausreichend Zeit, um die notwendigen Daten und Ressourcen im Hintergrund zu laden.
    \item \textbf{Navigation:} Es ist entscheidend, dass nach Ablauf der vier Sekunden die Benutzer zur Startseite (Homepage) der App navigiert werden und nicht zu anderen Seiten wie der Profil- oder Chatseite.
\end{itemize}

\section{Gestaltung des Ladebildschirms}

Für die Gestaltung des Ladebildschirms waren mehrere Schritte notwendig:

\subsection{Logodesign}

\begin{itemize}
    \item Das Logo wurde mit Canva erstellt und musste den Charakter und die Zielgruppe der App widerspiegeln. Es wurde darauf geachtet, dass das Design für eine SportTalentHub-App geeignet ist.
    \item Die Inspiration für das Logo wurde aus verschiedenen Quellen, wie der NFL, gezogen. Die Farben und das Design sollten sportlich und ansprechend sein.
    \item Nach der Erstellung wurde das Logo transparent gemacht, um es optimal in den Ladebildschirm integrieren zu können.
\end{itemize}

\subsection{Präsentation und Feedback}

\begin{itemize}
    \item Bevor das Logo endgültig in die App integriert wurde, wurde es im Rahmen eines wöchentlichen Meetings präsentiert. Das positive Feedback der Teammitglieder bestätigte die Eignung des Logos, sodass es anschließend in den Ladebildschirm eingefügt wurde.
\end{itemize}

\subsection{Technische Umsetzung}

\begin{itemize}
    \item Das Logo wurde im Ladebildschirm implementiert und ein animiertes Symbol hinzugefügt, das die Ladebewegung anzeigt.
    \item Es wurde darauf geachtet, dass nach dem Ablauf der vier Sekunden der Nutzer zur Startseite navigiert wird. Dies wurde durch entsprechende Programmierung in Flutter sichergestellt.
\end{itemize}

\section{Änderungen am App-Logo}

Neben der Erstellung und Implementierung des Logos für den Ladebildschirm war es auch notwendig, das App-Logo selbst zu ändern:

\subsection{Einfacher Prozess}

\begin{itemize}
    \item Das Ändern des App-Logos war technisch weniger anspruchsvoll. Es erforderte lediglich, das neue Logo in den entsprechenden Bereich des App-Projekts einzufügen.
\end{itemize}

\subsection{Integration in die Startseite}

\begin{itemize}
    \item Wir entschieden uns außerdem, das Logo auch auf der Startseite der App anzuzeigen. Hierzu wurde ein Code in die Startseite geschrieben, der diese Funktion realisiert.
\end{itemize}