\chapter{Ziele}
\section{Projektbegründung}
Mit der STH-App soll eine neuartige und benutzerfreundliche Plattform entwickelt werden, die den Austausch von Sportlern und Sportmanagern verbessert.
Es ist unverzichtbar, in einer immer stärker digitalisierten und vernetzten Welt effiziente Kommunikationswege und wirksame Instrumente bereitzustellen, um Talente zu identifizieren und zu fördern.
Das Ziel der STH-App ist es, diese Lücke zu überbrücken und einen wichtigen Ort für die Interaktion zwischen Spielern und Managern zu schaffen.

\noindent
Eines der zentralen Ziele der STH-App besteht darin, das Networking und den Austausch zwischen Sportlern und Sportmanagern zu fördern.
Die Implementierung einer plattformunabhängien App ermöglicht eine Kommunikation über geografische und organisatorische Grenzen hinweg.
Sportler haben die Möglichkeit, ihre Profile und Leistungen vorzustellen, während Sportmanager gezielt nach Talenten suchen und diese kontaktieren können.
Dadurch wird nicht nur die berufliche Laufbahn der Sportler gefördert, sondern es wird auch Sportmanagern ermöglicht, begabte Spieler zu finden und zu gewinnen.
\newline
Die weiteren Ziele, aufgeteilt in Muss- / Soll- / Kann- und Nicht-Ziele werden im nachfolgenden Punkt aufgelistet.

\section{Aufstellung der einzelnen Ziele}
\textbf{Muss-Ziele}
\begin{itemize}
    \item Die STH-App muss als plattformunabhängige App mit einem Source-Code implementiert werden, die auf verschiedenen Betriebssystemen und Geräten implementiert (vorzugsweise macOS und Windows) und ausgeführt (vorzugsweise iOS und Android) werden kann.
    \item Die Chat-Funktion in der STH App muss es ermöglichen, eine sichere Kommunikation zwischen den App-Nutzern aufbauen zu können.
    \item Die Projektdokumentation muss bis zum 21.06.2024 vom Projektteam komplett fertiggestellt und durch die anderen Projektmitglieder Korrektur gelesen sein.
    \item Das Projektteam muss bis 15.05.2024 einen fertigen MVP incl. aller Screens in der App bereitstellen.
    \item Die STH-App muss für jeden Screen ein einheitliches Design aufweisen und bis zur Präsentation Design-Anpassungen und Verbesserungen durch das Projektteam durchgeführt werden.
\end{itemize}

\textbf{Soll-Ziele}
\begin{itemize}
    \item Die STH-App soll bis zur Abgabe der Projektarbeit am 06.07.2024 auf verschiedenen Test-Geräten und Simulatoren / Emulatoren getestet werden und Fehler als neue Aufgaben eingestellt werden.
    \item Gefundene Fehler bei App-Tests sollen bis zum 06.07.2024 durch das Projektteam gelöst werden, um das Projekt gut und mit einer funktionierenden App präsentieren zu können.
    \item Für die Veröffentlichung sollen die datenschutzrechtliche Themen und Datensicherheit durch weitere Implementierungen oder das Ausweisen wichtiger Dokumente durch das Projektteam gewährleistet sein.
\end{itemize}

\textbf{Kann-Ziele}
\begin{itemize}
    \item Nach Abschluss aller Implementierungsschritte und ausreichenden Tests kann die STH App im Apple App Store oder Google Play Store veröffentlicht werden.
    \item Bis zu einer offiziellen App Veröffentlichung kann eine Login- und Registrierungsfunktion implementiert werden, die es ermöglicht, nutzerspezifische Inhalte zu speichern, abrufen und filtern zu können.
    \item Bei regelmäßigen Updates der App kann eine weitere Funktion für ein Hilfe- und Supportcenter angeboten werden.
\end{itemize}

\textbf{Nicht-Ziele}
\begin{itemize}
    \item Auf dem Home-Screen darf es keine Möglichkeit geben, Posts oder Feeds zu kommentieren.
    \item Stories und Reels ähnlich wie bei der Social Media App Instagram sollen nicht möglich sein.
    \item Die STH App soll nicht dazu dienen, Sportmanager bewerten zu können oder Transaktionen bzw. Verträge abzuschließen, sondern als Kommunikationsplattform zwischen Spieler und Sportmanager.
    \item Es dürfen keine Kosten für Spieler oder Sportmanager entstehen, um die App mit allen Funktionen nutzen zu können.
    \item Es dürfen keine laufenden Kosten wie Lizenzkosten für das Projektteam entstehen.
\end{itemize}

\section{Methodik}
Die STH-App nutzt agile Methoden, speziell Scrum, des Projektmanagements, die speziell auf die Projektanforderungen zugeschnitten sind.
Um sicherzustellen, dass die STH-App den Grundsätzen des kontinuierlichen Verbesserungsprozesses entspricht und auf neue Anforderungen von Interessengruppen flexibel reagieren kann, wurde die Scrum-Methodik gewählt.
Das STH-App-Projekt ist dabei in verschiedene Sprints unterteilt.
Für das STH-App-Projekt soll jeder Sprint eine Dauer von 2 Wochen haben.
Am Ende eines Sprint wird die entstandene Funktion bewertet.
Zu Beginn jedes Sprints findet das Sprint-Planning statt, um die Ziele und Aufgaben für den kommenden Sprint festzulegen.
Zunächst werden die anstehenden Aufgaben im Product Backlog Reihenfolge des Product Backlogs, welches alle anstehenden Aufgaben und Features enthält.
Während der Sprint-Planung werden die Aufgaben identifiziert, die im nächsten Sprint erledigt werden sollen. Das Team analysiert die benötigte Zeit für jede spezifische Aufgabe.
Während des Sprints finden regelmäßige Stand-up-Meetings statt, die als Weeklys bekannt sind. Hier berichtet jedes Teammitglied kurz über seine Arbeit, erreichte Fortschritte und bestehende Herausforderungen.
These regular meetings promote interaction and cooperation within the team, aiding in early identification and resolution of problems.
Nach jedem Sprint findet ein Sprint-Review statt, bei dem die Ergebnisse des Sprints präsentiert und bewertet werden.
Das Team informiert die Stakeholder über die durchgeführten Funktionen und Aufgaben und gibt Feedback.
This feedback is for updating the Product Backlog and planning the next steps.
Nach der Überprüfung des Sprints findet die Sprint-Retrospektive statt, in der das Team den vergangenen Sprint reflektiert und potenzielle Verbesserungen identifiziert.
In der Retrospektive werden positive Ereignisse, aufgetretene Probleme und mögliche Verbesserungen erörtert, um kontinuierlich am Entwicklungsprozess zu arbeiten.
Der Entwicklungsprozess beinhaltet mehrere Etappen, die mit der Anforderungsanalyse starten und detaillierte Beschreibungen der funktionalen und nicht-funktionalen Anforderungen bereitstellen.
\section{Technologie / Theorie / Hintergrund Beschreibung}
Die STH-App verwendet Firebase und Flutter. Das sind fortschrittliche Technologien und Frameworks, die sicherstellen, dass eine gute Leistung und Benutzerfreundlichkeit aufgewiesen wird. Firebase wird als Backend-Service verwendet, um eine schnelle und sichere Datenverwaltung zu gewährleisten.
GitHub wird genutzt, um die Versionskontrolle und das zentrale Code-Repository sicherzustellen. Visual Studio Code ist die verwendete Entwicklungsumgebung (IDE), die die Produktivität von Entwicklern steigert, indem sie zahlreiche Erweiterungen und Tools, vor allem für die verwendeten Frameworks, bietet.
Die erfolgreiche Umsetzung der STH-App erfordert die Anwendung agiler Entwicklungsmethoden und moderner Technologien. Durch die Verwendung agiler Prinzipien, moderner Entwicklungstools und theoretischer Erkenntnisse aus verschiedenen Fachbereichen wird eine effiziente, benutzerfreundliche und leistungsstarke Plattform geschaffen, die den Austausch zwischen Managern und Sportlern ermöglicht.