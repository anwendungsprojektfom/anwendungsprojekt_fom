\newpage
%reset pagestyle to continue with arabic numbering
\pagestyle{plain}
%pagenumbering
\fancypagestyle{plain}{%
	\fancyhf{} % clear all header and footer fields
	\fancyhead{} % clear all header fields
	\fancyfoot{} % clear all footer fields
	\fancyhead[C]{\textcolor{gray}\thepage}
	\renewcommand{\headrulewidth}{0pt}
	\renewcommand{\footrulewidth}{0pt}
}

\pagenumbering{Roman}

\setcounter{page}{5}

\section*{\fontsize{20}{0} \selectfont Abkürzungsverzeichnis}
\begin{acronym}[]\itemsep0pt %der Parameter in Klammern sollte die längste Abkürzung sein. Damit wird der Abstand zwischen Abkürzung und Übersetzung festgelegt
	\acro{SAE}{\textit{Society of Automotive Engineers}}
	\acro{kmh}[km/h]{Kilometer pro Stunde}
	\acro{t}{Tonnen}
	\acro{NO}{Stickoxide}
	\acro{HC}{unverbrannte Kohlenwasserstoffe}
	\acro{Wasser}[H\textsubscript{2}O]{Wasser}
	\acro{CO2}[CO\textsubscript{2}]{Kohlenstoffdioxid}
	\acro{CO}{Kohlenmonoxid}

\end{acronym}
\addcontentsline{toc}{chapter}{Abkürzungsverzeichnis}
\newpage