\chapter{Teilprojekte und Arbeitspakete}
Dieses Kapitel konzentriert sich auf die Arbeitspakete, die für die Entwicklung der STH-App von entscheidender Bedeutung sind.
Es beschreibt zunächst genau, welche Aufgabenstellung zu bearbeiten ist, inklusive der präzisen Definition und Strukturierung der Arbeitspakete sowie ihrer Integration in den Projektablauf. 
Dabei werden sowohl die Anforderungen der Arbeitspakete als auch die Herangehensweise mit Blick auf die Projektarbeit näher betrachtet, um ein umfassendes Verständnis für den Verlauf dieses Projekts zu gewährleisten.
Anschließend werden diese Arbeitspakete in einer tabellarischen Struktur zusammengefasst, die als Grundlage für die Meilensteinplanung dienen soll.
\\
\\
\\
\textbf{Definition der Arbeitspakete:} \\

\textbf{Arbeitspacket 1 vom 08.03.2024 bis 11.03.2024}
\begin{itemize}[itemsep=0pt]
	\item{Bezeichnung: }
	\item{Verantwortungsbereich: } 
	\item{Ressourcen: } 
    \item{Aufgabenstellung: laksdjh laskjd lkasjd oiasu jdoiasjd oiasjd oiasjd oiasjd aosidj oaisjd oiasjd oaisjd}
\end{itemize} 

\textbf{Arbeitspacket 2 vom 08.03.2024 bis 11.03.2024}
\begin{itemize}[itemsep=0pt]
	\item{Bezeichnung: }
	\item{Verantwortungsbereich: } 
	\item{Ressourcen: } 
    \item{Aufgabenstellung: oiashjd kjashd kjashd kashd kjash dkjashd asoiud aosijd oasiud oiasdas}
\end{itemize}

\textbf{Arbeitspacket 3 vom 08.03.2024 bis 11.03.2024}
\\
\hspace*{0,25cm}
\begin{tabular}{|l|p{8cm}|}
	\hline
	{Bezeichnung:} & \\
	\hline
	{Verantwortungsbereich:} & \\
	\hline
	{Ressourcen:} & \\
	\hline
	{Aufgabenstellung:} & {bdkjash dijashd oiasd oiasjd oih woiu eu iuas doasi dpoaskjd oiasdh asoidh }\\
	\hline
\end{tabular}

\begin{table}[h]
	\caption{Tabellarische Auflistung der Arbeitspakete}
	\centering
	\resizebox{\columnwidth}{!}{
	\begin{tabular}{|c|l|l|}
		\hline
		\textbf{Arbeitspaket vom: } & \textbf{Beschreibung} & \textbf{Dauer} \\
		\hline
		\hline
		{08.03.2024} & {1: Einrichtung der Entwicklungsumgebung für Flutter} & {3 Tage} \\
		\hline
		{08.03.2024} & {2: Einrichtung der Entwicklungsumgebung für LaTeX} & {3 Tage} \\
		\hline
		{08.03.2024} & {3: Erstellung des Repositorys auf GitHub} & {1 Tag} \\
		\hline
		{13.03.2024} & {4: Definition der App Funktionalitäten} & {1 Tag} \\
		\hline
		{14.03.2024} & {5: Erstellung der ersten Projektstruktur in Visual Studio Code/ Flutter} & {1 Tag} \\
		\hline
        {16.03.2024} & {6: Implementierung des Homes-Screens inklusive Menüleiste am unteren Bildschirmrand} & {1 Woche} \\
		\hline
        {18.03.2024} & {7: Implementierung des Chat-Screens} & {2 Wochen} \\
		\hline
        {21.03.2024} & {8: Implementierung des Accountprofile-Screens inklusive Avatar und personenbezogene Daten} & {2 Wochen} \\
		\hline
        {23.03.2024} & {9: Implementierung des Search-Screens} & {1 Woche} \\
		\hline
		{16.03.2024} & {10: Implementierung einer globalen Menüleiste am unteren Bildschirmrand für alle Pages} & {1 Woche} \\
		\hline
        {24.03.2024} & {11: Backend-Initialisierung (Firebase)} & {2 Wochen} \\
		\hline
        {24.03.2024} & {12: Implementierung eines Bearbeitungsbuttons für den Accountprofile-Screen} & {1 Woche} \\
		\hline
        {24.03.2024} & {13: Gestaltung eines Logos für die STH-APP} & {1 Woche} \\
		\hline
        {25.03.2024} & {14: Implementierung einer Ladebildschirm-Funktion beim Start der STH-App} & {1 Woche} \\
		\hline
        {27.03.2024} & {15: Implementierung von GetStream-Chat} & {1 Woche} \\
		\hline
        {27.03.2024} & {16: Anpassung der App-Leiste auf dem Chat-Bildschirm} & {1 Tag} \\
		\hline
        {29.03.2024} & {17: Implementierung eines Ladebildschirms mit einem Logo} & {3 Tage} \\
		\hline
        {31.03.2024} & {18: Implementierung der Backend-Konfiguration (Firebase) und deren Methoden} & {1 Woche} \\
		\hline
        {01.04.2024} & {19: Implementierung einer benutzerdefinierten Seitenroute beim Wechsel von den Pages} & {1 Woche} \\
		\hline
        {02.04.2024} & {20: Anpassung des Chat-Screens und der benutzerdefinierte App-Leiste} & {1 Woche} \\
		\hline
        {02.04.2024} & {21: Anpassung des Chanels im Chat-Screen} & {1 Woche} \\
		\hline
        {02.04.2024} & {22: Dokumentationsstruktur in Latex aktualisiert} & {1 Tag} \\
		\hline
        {03.04.2024} & {23: Implementierung eines lokalen Speichers für den Accountprofile-Screen} & {1 Woche} \\
		\hline
        {03.04.2024} & {24: Implementierung einer Funktion zum Speichern/Laden für den Accountprofile-Screen} & {1 Woche} \\
		\hline
        {03.04.2024} & {25: Implementierung von RegEx-Richtlinien in Bezug auf personenbezogene Daten} & {1 Woche} \\
		\hline
        {03.04.2024} & {26: Implementierung einer Hovering-Funktion für personenbezogene Daten} & {1 Woche} \\
		\hline
        {03.04.2024} & {27: Implementierung des Profile-Screens mit Avatar, Bildern und Mediathek} & {2 Wochen} \\
		\hline
        {10.04.2024} & {28: Implementierung von Funktionen zum Auswählen und Hochladen von Bildern und Videos} & {1 Woche} \\
		\hline
        {10.04.2024} & {29: Implementierung eines lokalen Speichers für den Profile-Screen} & {1 Woche} \\
		\hline
        {10.04.2024} & {30: Implementierung einer Funktion zum Speichern/ Laden für den Profile-Screen} & {1 Woche} \\
		\hline
        {10.04.2024} & {31: Anpassung der App-Leiste für die Navigation und das Verhalten beim Seitenwechsel} & {1 Tag} \\
		\hline
        {13.04.2024} & {32: Verwendung von Shared Preferences zusammen mit Firebase-Testdaten} & {1 Woche} \\
		\hline
        {14.04.2024} & {33: Verlinkung der Pages von Profile-Screen und Accountprofile-Screen} & {1 Tag} \\
		\hline
        {16.04.2024} & {34: Aktulalierung des Loadingscreens} & {1 Tag} \\
		\hline
        {16.04.2024} & {35: Hinzufügen vom Logo auf die Startseite} & {1 Woche} \\
		\hline
        {21.04.2024} & {36: Implementierung eines Buttons für das hochladen von Bildern und Videos} & {1 Woche} \\
		\hline
        {21.04.2024} & {37: Implementierung von Funktionen zur Übertragung vom Avatar an das Backend} & {2 Wochen} \\
		\hline
        {25.04.2024} & {38: Implementierung von Funktionen zur Übertragung von Bildern und Videos an das Backend} & {1 Woche} \\
		\hline
        {27.04.2024} & {39: Anpassung der Seiten-Navigation und Entfernung von Animationen beim Wechsel} & {1 Tag} \\
		\hline
		{30.04.2024} & {40: Update username form Accountprofile-Screen} & {1 Woche} \\
		\hline
	\end{tabular}
	}
\end{table}

