\newpage
\chapter{Produktanforderungen und technische Architektur}

In diesem Kapitel werfen wir einen umfassenden Blick auf die Produktanforderungen und die technische Architektur der STH-App.
Die STH-App bietet weit mehr als nur das Teilen von Inhalten. Sie fungiert als Social-Media-Plattform, auf der Sportmanager und Sportler miteinander kommunizieren, Multimedia-Inhalte, Fähigkeiten sowie personenbezogene Profildaten teilen können. Daher ist es von entscheidender Bedeutung, dass die Produktanforderungen einzelner Funktionen und Features separat zur Implementierung bereitgestellt werden, um den unterschiedlichen Anforderungen und Erwartungen beider Gruppen gerecht zu werden.
Genau genommen wird die STH-App eine Vielzahl von Funktionen bieten, mit denen Benutzer ihre sportlichen Ziele verfolgen und sich dabei mit anderen Benutzern vernetzen können.
Die Hauptbildschirme der STH-App, wie beispielsweise der Home-Screen, Chat-Screen, Search-Screen, Profile-Screen und der Loading-Screen, sind interaktive Schnittstellen, die die Sporterfahrung sowohl für Manager als auch für Sportler bereichern.
Der Home-Screen der STH-App bietet einen umfassenden Überblick über die sportlichen Aktivitäten und Leistungen durch die Profilanlagen der jeweiligen Benutzer. Auf diese Weise können Manager beispielsweise die Profile ihrer Athleten verfolgen und die darin enthaltenen Daten begutachten. Gleichzeitig ermöglicht er den Benutzern, ihre Ziele zu setzen, ihre Fähigkeiten zur Schau zu stellen und sich mit anderen Athleten zu vernetzen.
Der Chat-Screen hingegen bietet für Sportmanager und Athleten einen zentralen Bereich für den direkten Austausch von Textnachrichten über Chatkanäle. Dies ermöglicht eine reibungslose Kommunikation und Zusammenarbeit, indem wichtige Informationen schnell und effizient ausgetauscht werden können, um die gemeinsamen Vereinbarungen und Ziele zu erreichen.
Auf dem Search-Screen können Benutzer nach spezifischen personenbezogenen Daten oder Hashtags suchen. So wird beispielsweise das Auffinden neuer Benutzer oder das Suchen nach individuellen Informationen erleichtert, die den eigenen Interessen entsprechen.
Beim Profile-Screen können Benutzer die digitale Visitenkarte eines jeden Sportlers oder Managers einsehen. Diese enthält personenbezogene Daten sowie eine Beschreibung der Person, Bilder, Videos und Qualifikationen und ermöglicht es den Benutzern, sich ein umfassendes Bild von anderen Nutzern zu machen und potenzielle Verbindungen zu knüpfen oder Zusammenarbeit zu erleichtern.
Die Leistung der STH-App spielt eine ebenso wichtige Rolle, um sicherzustellen, dass die Benutzer ein reibungsloses Erlebnis genießen können. Dabei erfordert es eine effiziente Programmierung und Optimierung sowohl auf dem Frontend als auch auf dem Backend der App. Der Loading-Screen unterstützt dabei, sämtliche Initialisierungsfeatures der STH-App aufzurufen und damit alle verbundenen Services zu starten und zu Initialisieren.
Neben den Produktanforderungen ist auch die technische Architektur der App von entscheidender Bedeutung. Hierbei spielt die technische Architektur einer App eine entscheidende Rolle, da sie die Grundlage für Effizienz und Skalierbarkeit bildet.
Ein sorgfältig ausgewähltes Set von Technologien und Plattformen ist daher unerlässlich, um sicherzustellen, dass die Anwendung nicht nur reibungslos funktioniert, sondern auch die sich ständig wandelnden Anforderungen der Benutzer erfüllen kann.
Als Entwicklungsplattform dient Visual Studio Code, eine leistungsstarke IDE, die speziell für die Entwicklung von Flutter-Apps optimiert ist. Visual Studio Code bietet eine Fülle von Funktionen und Erweiterungen, die den Entwicklungsprozess beschleunigen und vereinfachen.
Um mit der Entwicklung in Flutter plattformunabhängig zu gestalten, ist es neben Visual Studio Code und dem Flutter-Framework wichtig, auch Android Studio zu installieren. Dies gewährleistet, dass Entwickler sowohl für iOS als auch für Android entwickeln können. Während das Flutter-Framework eine benutzerfreundliche Umgebung und eine Vielzahl von Smartphone-Emulatoren für die Entwicklung von Flutter-Apps bietet, ist Android Studio unverzichtbar für die spezifische Entwicklung von Android-Apps. Durch die Kombination beider Komponenten können Entwickler effizient und reibungslos plattformübergreifend arbeiten und damit sicherstellen, dass ihre Anwendungen auf verschiedenen Geräten und Betriebssystemen laufzeitfähig initialisiert werden können.
Darüber hinaus ist die Verwendung von Visual Studio Code in Kombination mit Basic-Miktex und Strawberry-Perl für die Erstellung von LaTeX-Dokumenten von entscheidender Bedeutung. Diese Tools ermöglichen nicht nur eine effiziente Bearbeitung und Kompilierung von LaTeX-Dokumenten, sondern auch eine reibungslose Integration dank des Docker-Add-Ons, das eine containerisierte Entwicklungsumgebung bereitstellt. Das Addon ermöglicht in diesem Zusammenhang eine verbesserte Portabilität und Konsistenz des LaTeX-Entwicklungsprozesses für die Dokumentation.
Für die Versionskontrolle und Zusammenarbeit wird auf GitHub gesetzt, eine führende Plattform für die Zusammenarbeit an Softwareprojekten. Hier können Entwickler gemeinsam am Code arbeiten, Änderungen verfolgen und Projekte dokumentieren.
Die klare Definition der Produktanforderungen sowie die Festlegung der Arbeitspakete über GitHub-Commits, in denen die Aufgabenstellung detailliert beschrieben wird, bilden zusammen mit der soliden technischen Architektur das Fundament für eine effektive Zusammenarbeit in der Entwicklung und Dokumentation des Projekts.
Flutter selbst fungiert als Cross-Platform-Framework für die Entwicklung sowohl des Frontends als auch des Backends der STH-App. Die Verwendung von Flutter bietet eine Reihe von Vorteilen, darunter eine hohe Leistung, schnelle Entwicklung und einfache Wartung.
Für das Backend der STH-App wird Firebase genutzt, eine Plattform von Google, die eine breite Palette von Diensten für die Entwicklung von Web- und Mobile-Apps bietet. Mit Funktionen wie Echtzeitdatenbanken, Authentifizierungsdiensten und Cloud-Speicher bietet Firebase eine robuste Lösung für die Anforderungen der STH-App.
