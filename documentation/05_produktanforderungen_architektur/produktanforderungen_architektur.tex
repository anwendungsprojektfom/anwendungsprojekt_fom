\newpage
\chapter{Produktanforderungen und technische Architektur}

In diesem Kapitel werfen wir einen umfassenden Blick auf die Produktanforderungen und die technische Architektur der STH-App.
Die STH-App bietet weit mehr als nur das Teilen von Inhalten. Sie fungiert als Social-Media-Plattform, auf der Sportmanager und Sportler miteinander kommunizieren und Fähigkeiten sowie personenbezogene Profildaten teilen können. Daher ist es von entscheidender Bedeutung, dass die Produktanforderungen einzelner Funktionen und Features separat zur Implementierung bereitgestellt werden, um den unterschiedlichen Anforderungen und Erwartungen beider Gruppen gerecht zu werden.
Insbesondere mit der großen Anzahl an Funktionen ermöglicht die STH-App den Nutzern, ihre sportliche Karriere zu teilen und sich dabei mit anderen Benutzern vernetzen können.
Der Home-Screen der STH-App bietet einen umfassenden Überblick über die sportlichen Aktivitäten und Leistungen durch Posts der jeweiligen Benutzer. Auf diese Weise können Manager beispielsweise die Profile von Athleten verfolgen. Gleichzeitig ermöglicht er den Sportlern, ihre Fähigkeiten zu veröffentlichen und sich mit anderen Athleten und Sportmanagern anderer Sportvereine zu vernetzen.
Der Chat-Screen hingegen bietet für Sportmanager und Athleten einen zentralen Bereich für den direkten Austausch von Textnachrichten über Chatkanäle. Dies ermöglicht eine reibungslose Kommunikation und Zusammenarbeit, indem wichtige Informationen schnell und effizient ausgetauscht werden können, um die gemeinsamen Vereinbarungen und Ziele zu erreichen.
Auf dem Search-Screen können Benutzer nach spezifischen personenbezogenen Daten oder Hashtags suchen. So wird beispielsweise das Auffinden neuer Benutzer oder das Suchen nach individuellen Informationen erleichtert, die den eigenen Interessen entsprechen.
Beim Profile-Screen können Benutzer die digitale Visitenkarte eines jeden Sportlers oder Managers einsehen. Diese enthält personenbezogene Daten sowie eine Beschreibung der Person, Bilder, Videos und Qualifikationen und ermöglicht es den Benutzern, sich ein umfassendes Bild von anderen Nutzern zu machen und potenzielle Verbindungen zu knüpfen oder Zusammenarbeit zu erleichtern.
Die Leistung der STH-App spielt eine ebenso wichtige Rolle, um sicherzustellen, dass die Benutzer ein reibungsloses Erlebnis genießen können. Dabei erfordert es eine effiziente Programmierung und Optimierung sowohl auf dem Frontend als auch auf dem Backend der App. Der Loading-Screen unterstützt dabei, sämtliche Initialisierungsfeatures der STH-App aufzurufen und damit alle verbundenen Services zu starten und zu Initialisieren.
Neben den Produktanforderungen ist auch die technische Architektur der App von entscheidender Bedeutung. Hierbei spielt die technische Architektur einer App eine entscheidende Rolle, da sie die Grundlage für Effizienz und Skalierbarkeit bildet.
Ein sorgfältig ausgewähltes Set von Technologien und Plattformen ist daher unerlässlich, um sicherzustellen, dass die Anwendung nicht nur reibungslos funktioniert, sondern auch die sich ständig wandelnden Anforderungen der Benutzer erfüllen kann.
Als Entwicklungsplattform dient Visual Studio Code, eine leistungsstarke IDE, die speziell für die Entwicklung von Flutter-Apps optimiert ist. Visual Studio Code bietet eine Fülle von Funktionen und Erweiterungen, die den Entwicklungsprozess beschleunigen und vereinfachen.
Die klare Definition der Produktanforderungen sowie die Festlegung der Arbeitspakete über GitHub-Commits, in denen die Aufgabenstellung detailliert beschrieben wird, bilden zusammen mit der soliden technischen Architektur das Fundament für eine effektive Zusammenarbeit in der Entwicklung und Dokumentation des Projekts.
Flutter selbst fungiert als Cross-Platform-Framework für die Entwicklung des Frontends als auch der Schnittstelle zum Backend der STH-App. Die Verwendung von Flutter bietet eine Reihe von Vorteilen, darunter eine hohe Leistung, schnelle Entwicklung und einfache Wartung.
Für das Backend der STH-App wird Firebase genutzt, eine Plattform von Google, die eine breite Palette von Diensten für die Entwicklung von Web- und Mobile-Apps bietet. Mit Funktionen wie Echtzeitdatenbanken, Authentifizierungsdiensten und Cloud-Speichern bietet Firebase eine robuste Lösung für die Anforderungen der STH-App.
