\chapter{Ziele}
In diesem Kapitel werden die Muss-, Soll-, Kann- und Nicht-Ziele für das Projekt STH-App aufgezeigt.
\\

\textbf{Muss-Ziele}
\begin{itemize}
    \item Die STH-App muss als plattformunabhängige App mit einen Source-Code implementiert werden, die auf verschiedenen Betriebssystemen und Geräten implementiert (vorzugsweise macOS und Windows) und ausgeführt (vorzugsweise iOS und Android) werden kann.
    \item Die Chat-Funktion in der STH App muss es ermöglichen, eine sichere Kommunikation zwischen den App-Nutzern aufbauen zu können.
    \item Die Projektdokumentation muss bis zum 21.06.2024 vom Projektteam komplett fertiggestellt und durch die anderen Projektmitglieder Korrektur gelesen sein.
    \item Das Projektteam muss bis 15.05.2024 einen fertigen MVP incl. aller Screens in der App bereitstellen.
    \item Die STH-App für jeden Screen ein einheitliches Design aufweisen und bis zur Präsentation Design-Anpassungen und Verbesserungen durch das Projektteam durchgeführt zu haben.
\end{itemize}

\textbf{Soll-Ziele}
\begin{itemize}
    \item Die STH-App soll bis zur Abgabe der Projektarbeit am 06.07.2024 auf verschiedenen Test-Geräten und Simulatoren / Emulatoren getestet werden und Fehler als neue Aufgaben eingestellt werden.
    \item Gefundene Fehler bei App-Tests sollen bis zum 06.07.2024 durch das Projektteam gelöst werden, um das Projekt gut und mit einer funktionierenden App präsentieren zu können.
    \item Für die Veröffentlichung sollen die datenschutzrechtliche Themen und Datensicherheit durch weitere Implementierungen oder das Ausweisen wichtiger Dokumente durch das Projektteam gewährleistet sein.
\end{itemize}

\textbf{Kann-Ziele}
\begin{itemize}
    \item Nach Abschluss aller Implementierungsschritte und ausreichenden Tests kann die STH App im Apple Appstore oder Google Play Store veröffentlicht werden.
    \item Bis zu einer offiziellen App Veröffentlichung kann eine Login- und Registrierungsfunktion implementiert werden, die es ermöglicht, nutzerspezifische Inhalte zu speichern, abrufen und filtern zu können.
    \item Bei regelmäßigen Updates der App kann eine weitere Funktion für ein Hilfe- und Supportcenter angeboten werden.
\end{itemize}

\textbf{Nicht-Ziele}
\begin{itemize}
    \item Auf dem Homescreen darf es keine Möglichkeit geben, Posts oder Feeds zu kommentieren.
    \item Stories und Reels ähnlich wie bei der Social Media App Instagram sollen nicht möglich sein.
    \item Die STH App soll nicht dazu dienen, Sportmanager bewerten zu können oder Transaktionen bzw. Verträge abzuschließen sondern als Kommunikationsplattform zwischen Spieler und Sportmanager.
    \item Es dürfen keine Kosten für Spieler oder Sportmanager entstehen, um die App mit allen Funktionen nutzen zu können.
    \item Es dürfen keine laufenden Kosten wie Lizenzkosten für das Projektteam entstehen.
\end{itemize}