\chapter{Projektressourcen}
Im nächsten Schritt werden die Ressourcen beschrieben. Dazu zählen die am STH-App-Projekt beteiligten Personen und die materiellen Ressourcen.
Da es sich um ein Studienprojekt handelt, das die Entwicklung eines MVP einer App zum Ziel hatte, sind die aufgeführten Ressourcen speziell auf diesen Rahmen zugeschnitten.

\section{Personen}
\subsection{Projektteam}
Das Projektteam bestand aus vier Mitgliedern: Jonas Waigel, Fitim Makolli, Hasan Deveci und Vatsegkan Zournatsidis.
Jonas Waigel wurde als Projektleiter gewählt und war für die Koordination und Leitung des Teams verantwortlich.
Alle Teammitglieder brachten ihre individuellen Fähigkeiten und Erfahrungen ein, um gemeinsam die STH-App zu entwickeln.

\subsection{Personalaufwand}
Der Personalaufwand wurde in Personentagen gemessen. Da das Projekt über einen Zeitraum von vier Monaten lief, wurde der wöchentliche Aufwand pro Teammitglied auf etwa 10 Stunden geschätzt, was insgesamt 40 Stunden pro Woche für das gesamte Team ergibt. 
Über die Projektlaufzeit hinweg summiert sich dies auf etwa 2.560 Stunden (40 Stunden/Woche × 15 Wochen).


\subsection{Personalkosten}

\begin{itemize}
    \item \textbf{Jonas Waigel}
    \begin{itemize}
        \item \textbf{Stundensatz:} 120,00 €
        \item \textbf{Gesamtstunden:} 280 Stunden
        \item \textbf{Zeitraum:} 08.03.2024 - 18.06.2024
        \item \textbf{Gesamtkosten:} 33.600,00 €
    \end{itemize}

    \item \textbf{Fitim Makolli}
    \begin{itemize}
        \item \textbf{Stundensatz:} 110,00 €
        \item \textbf{Gesamtstunden:} 350 Stunden
        \item \textbf{Zeitraum:} 08.03.2024 - 18.06.2024
        \item \textbf{Gesamtkosten:} 38.500,00 €
    \end{itemize}

    \item \textbf{Hasan Deveci}
    \begin{itemize}
        \item \textbf{Stundensatz:} 130,00 €
        \item \textbf{Gesamtstunden:} 210 Stunden
        \item \textbf{Zeitraum:} 08.03.2024 - 18.06.2024
        \item \textbf{Gesamtkosten:} 27.300,00 €
    \end{itemize}

    \item \textbf{Vatsegkan Zournatsidis}
    \begin{itemize}
        \item \textbf{Stundensatz:} 150,00 €
        \item \textbf{Gesamtstunden:} 420 Stunden
        \item \textbf{Zeitraum:} 08.03.2024 - 18.06.2024
        \item \textbf{Gesamtkosten:} 63.000,00 €
    \end{itemize}
\end{itemize}

\textbf{Gesamtkosten:} 162.400,00 €

\section{Ressourcen}
\subsection{Ressourcenliste}
Die materiellen Ressourcen, die für die Entwicklung der STH-App erforderlich waren, umfassten:
\begin{itemize}
    \item Hardware: Laptops/Computer für jedes Teammitglied
    \item Software: Entwicklungsumgebungen (Visual Studio Code, xCode und Android Studio), Versionskontrollsystem (GitHub)
    \item Plattformen: Firebase für das Backend, Flutter für die plattformübergreifende App-Entwicklung
    \item Kommunikationstools: Microsoft Teams für die teaminterne Kommunikation und für regelmäßige Meetings und Stand-ups
    \item Testgeräte: Smartphones (iOS und Android) und Simulatoren / Emulatoren von xCode und Android Studio für Testzwecke
\end{itemize}

\subsection{Ressourcenkosten}

\begin{itemize}
    \item Hardware: 4 Laptops à 1200 € = 4800 €
    \item Software/ Entwicklungsumgebungen: Kostenfrei (durch Bildungsprogramme oder vorhandene Lizenzen)
    \item Plattformen: Firebase (kostenlose Nutzung innerhalb von bestimmten Grenzen), Flutter (kostenlos)
    \item Kommunikationstools: Microsoft Teams (kostenlose Nutzung für Bildungseinrichtungen)
    \item Testgeräte: Nutzung vorhandener Smartphones und kostenloser Simulatoren/Emulatoren
\end{itemize}

\noindent
Insgesamt wurden alle benötigten Ressourcen effizient genutzt, um die Entwicklung des MVP der STH-App ohne zusätzliche finanzielle Belastungen zu realisieren.
Dies ermöglichte es dem Team, sich vollständig auf die technischen, funktionalen und prozessualen Aspekte der App-Entwicklung sowie auf die Methoden des Projektmanagements zu konzentrieren und den Prototypen innerhalb des vorgegebenen Zeitrahmens erfolgreich umzusetzen.
