\chapter{Fazit}
Rückblickend auf das Projekt STH-App kann man sagen, dass die Ziele, die zu Beginn des Projekts festgelegt wurden, erfüllt wurden.
Jetzt bietet die STH-App eine neue Möglichkeit, Sportmanager und Sportler über die Chat-Funktion in Kontakt zu bringen.
Darüber hinaus sorgt das verwendete Framework Flutter für eine plattformunabhängige Interaktion zwischen Android- und Apple iOS-Geräten sowie für regelmäßige Aktualisierung.
Das Design ist speziell für die STH-App konzipiert und zeichnet sich durch aussagekräftige Icons durch Konsistenz und eine gute Benutzererfahrung aus.
Die Einbindung der Google Firebase Tools und insbesondere die unmittelbare Datenübertragung wurden als erfolgreich angesehen.\newline
Der Einsatz moderner Tools und die sehr enge Zusammenarbeit der Teammitglieder waren die entscheidenden Faktoren für den aktuellen Stand der Anwendung.
Der Stand des Projekts wurde wesentlich durch die wöchentlichen strukturierten Besprechungen der aktuellen Fortschritte, der aufgetretenen Probleme bei den Teammitgliedern und der weiteren Planung erreicht.
Im Verlauf des Projekts kamen bei den Tests der App häufig Fehler auf und die Teammitglieder schlugen Verbesserungsvorschläge vor.
Dennoch wurden alle gesetzten Meilensteine erreicht, obwohl es anfangs technische Probleme beim Einrichten der Entwicklungsumgebungen auf den verschiedenen Betriebssystemen als auch Kompilierfehler aufgrund von Änderungen am Sourcecode und Entwicklungsverzögerungen gab.
Auf lange Sicht bietet die STH-App aufgrund ihrer unkomplizierten und zuverlässigen Bauweise sowie ihrer Benutzerfreundlichkeit ein beträchtliches Potenzial auf unterschiedlichen Märkten.\newline
Wir konnten im Rahmen des Projekts wichtige Erfahrungen sammeln, vor allem im Hinblick auf die agile Arbeitsweise und die Entwicklung von Anwendungen.
Auch das offene Feedback und die fortlaufende Verbesserung auf der Grundlage der App-Testergebnisse waren entscheidende Erfolgsfaktoren.
\chapter{Ausblick}

Um die STH-App weiterzuentwickeln und die Benutzererfahrung zu verbessern, sind folgende Erweiterungen und Verbesserungen vom Projektteam geplant.

\begin{itemize}
\item Verbesserung des Search-Screen-Algorithmus: Die Verbesserung des Suchalgorithmus führt zu einer Steigerung der Effizienz und Präzision der Suchergebnisse, was es den Nutzern ermöglicht, die gewünschten Informationen schneller und genauer zu finden.
\item In-App-Linkouts einrichten, um direkt auf einen anderen Bildschirm zu wechseln: Mit dieser Funktion können Benutzer direkt zwischen verschiedenen Bildschirmen wechseln. Dies trägt zur Verbesserung des Navigierens in der Anwendung bei.
\item Integration von KI-gestützten Funktionen: Die Verwendung von künstlicher Intelligenz ermöglicht es, auf die STH-App angepasste Empfehlungen und automatisierte Abläufe zu generieren, die die Benutzererfahrung noch individueller und effizienter machen. Zum Beispiel können KI-Funktionen auf dem Home-Screen oder Search-Screen dazu beitragen, Sportler oder Sportmanager zu finden.
\item Lokalisierung und Übersetzungen von Texten in der App: Durch die Lokalisierung und Übersetzung der Inhalte der App in unterschiedliche Sprachen wird die Verbreitungsfähigkeit der STH-App vergrößert und die Benutzerfreundlichkeit für internationale Anwender gesteigert.
\item Die Einführung von Personalisierungsmöglichkeiten: Es wird den Nutzern ermöglicht, die App individuell anzupassen, was zu einer Steigerung ihres Engagements und ihrer Zufriedenheit führt.
\end{itemize}
Darüber hinaus soll der Source-Code der STH-App mit technischen Versionsupdates des Frameworks Flutter und die dazugehörigen Dependencies aktualisiert werden sowie die Richtlinien von Apple und Google ebenfalls berücksichtigt werden. Dadurch wird sichergestellt, dass die Anwendung stets auf dem neuesten Stand der Technik bleibt und eine optimale Leistung erbringt.
Andere bedeutende Aspekte sind die Leistung der STH-App auf den Nutzergeräten sowie Datenschutz und Sicherheit. 
Das Projektteam wird in den nächsten Monaten vermehrt auf Marktanalysen und Innovationsprozesse setzen, um die App fortlaufend zu verbessern und das Nutzererlebnis zu verbessern. Dabei entsteht eine ganzheitliche Teststrategie, die unterschiedliche Testverfahren wie User Interface Tests, A/B-Tests, manuelle Tests, Testautomatisierung, Unit-Tests und Crowdtests beinhaltet. Dies geschieht, um die Wünsche und Rückmeldungen der Nutzer gezielt zu erfassen und eine Vielzahl von Nutzern zu berücksichtigen.
Auf lange Sicht plant das Projektteam, die STH-App mit Wartung und technischer Unterstützung zu veröffentlichen. Um fortlaufendes Feedback zu bekommen und regelmäßig Fehlerbehebungen sowie Verbesserungen durch App-Updates durchzuführen, soll eine Community oder ein Forum entwickelt werden. \newline
Zusammenfassend kann festgehalten werden, dass dieses Projekt in Zukunft viele Chancen und Möglichkeiten für den Erfolg der STH-App mit sich bringt. Die STH-App wird ihre Position als innovative und nützliche App für Sportmanager und Sportler weiter festigen und ausbauen.
