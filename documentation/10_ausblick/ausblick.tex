\chapter{Fazit}
Im Rückblick auf das Projekt STH-App lässt sich festhalten, dass die zu Projektbeginn definitierten Ziele erreicht wurden.
Die STH-App schafft nun eine innovative Lösung, Sportmanager und Sportler über die Chat-Funktion miteinander kommunizieren zu lassen.
Zudem gewährleistet das genutzte Framework die Plattformunabhängigkeit zwischen Apple iOS Geräten und Android Geräten und zusätzlich die Aktualität durch regelmäßige Updates.
Das Design wurde speziell für die STH-App entwickelt und zeigt Konsistenz und eine gute User Experience mithilfe von aussagekräftigen Icons.
Des Weiteren wurde die Integration der Google Firebase Tools und vor allem die sofortige Datenübertragung als gelungen betrachtet.\newline
Für diesen aktuellen Stand der App war vor allem die sehr enge Zusammenarbeit der Teammitglieder und der Einsatz von modernen Tools ausschlaggebend.
Speziell die wöchentlichen strukturierten Besprechungen mit den aktuellen Fortschritten und Problemen der Teammitglieder, aber auch die weitere Planung haben sehr geholfen, den aktuellen Stand des Projektes zu erreichen.
Während des Projektes sind mithilfe der manuellen kontinuierlichen App-Tests bei der Entwicklung neuer Funktionen von den Teammitgliedern regelmäßig Bugs und Verbesserungsvorschläge gefunden, die während den Weekly-Standups dokumentiert und dementsprechend eine neue Aufgabe im Teamboard erstellt wurde.\newline
Trotz anfänglicher technischer Schwierigkeiten beim Einrichten der Entwicklungsumgebungen auf den unterschiedlichen Betriebssystemen, Kompilierfehler durch Änderungen am Sourcecode und Verzögerungen bei der Entwicklung wurden dennoch alle gesetzten Meilensteine erreicht.
Langfristig kann die STH-App vor allem durch die einfache und zuverlässige Architektur und die Benutzerfreundlichkeit ein großes Potential in verschiedenen Märkten erreichen.\newline
Während des Projekts konnten wir wertvolle Erfahrungen sammeln, insbesondere in Bezug auf die agile Arbeitsweise und die Nutzung von Tools zur Zusammenarbeit.
Ein entscheidender Erfolgsfaktor war auch das offene Feedback und die kontinuierliche Verbesserung basierend auf den Ergebnissen der App-Tests.

\chapter{Ausblick}
Mit Blick auf die zukünfige Entwicklung der STH-App sind folgende Funktionen geplant:
\begin{itemize}
    \item Verbesserung des Suchalgorithmus für den Search-Screen
    \item Anpassungen des Designs und Designverbesserungen
    \item Verbesserung der User Experience durch Funktionsanpassungen 
    \item Einrichten von In-App-Linkouts für einen direkten Wechsel zu einem anderen Screen
    \item Integration von KI-basierten Features
    \item Lokalisierung und Übersetzungen von Texten in der App
    \item Implementierung von Personalisierungsoptionen
\end{itemize}
Zusätzlich sollen auch technische Versionsupdates des Frameworks Flutter, der dazugehörigen Dependencies und von Apple und Google berücksichtigt und der Source-Code der STH-App dementsprechend geupdated werden.
Weitere wichtige Punkte sind die Performance der STH-App auf den Geräten der Nutzer, die Sicherheit und der Datenschutz.
Die oben genannte Lokalisierung soll nach der Veröffentlichung in den App Stores auch zur Erschließung von neuen Märkten beitragen.
Mögliche Herausforderungen, wie die Anpassung an verschiedene regionale Anforderungen, werden durch gezielte Forschung und Entwicklung adressiert.
In den kommenden Monaten wird das Projektteam daher verstärkt auf Innovationsprozesse und Marktanalysen setzen, um die App kontinuierlich zu optimieren und das Benutzererlebnis zu verbessern.\newline
Um valides Testfeedback zu bekommen, arbeitet das Projektteam an einer Teststrategie, die verschiedene Testmethoden wie UI-Tests, A/B Testen, manuelle Tests, Testautomatisierung, Unit-Tests und Crowdtests enthält.
Das dient vor allem dazu, um Benutzerwünsche und -feedback gezielt zu erhalten und dabei ein breites Spektum an Nutzern berücksichtigt wird.\newline
Langfristig plant das Projektteam eine mögliche Veröffentlichung der STH-App inkl. Wartung und technischem Support.
Dazu soll eine Community bzw\. ein Forum aufgebaut werden und basierend auf dem Feedback regelmäßige Fehlerbehebungen über App-Updates zu bringen.\newline
Abschließend lässt sich feststellen, dass dieses Projekt in der Zukunft sehr viele Möglichkeiten und Optionen für den Erfolg der STH-App bietet.