\chapter{Fazit}
Im Rückblick auf das Projekt STH-App lässt sich festhalten, dass die zu Projektbeginn definierten Ziele erreicht wurden.
Die STH-App schafft nun eine innovative Lösung, Sportmanager und Sportler über die Chat-Funktion miteinander kommunizieren zu lassen.
Zudem gewährleistet das genutzte Framework Flutter die Plattformunabhängigkeit zwischen Apple iOS Geräten und Android Geräten und zusätzlich die Aktualität durch regelmäßige Updates.
Das Design wurde speziell für die STH-App entwickelt und zeigt Konsistenz und eine gute User Experience mithilfe von aussagekräftigen Icons.
Des Weiteren wurde die Integration der Google Firebase Tools und vor allem die sofortige Datenübertragung als gelungen betrachtet.\newline
Für diesen aktuellen Stand der App war vor allem die sehr enge Zusammenarbeit der Teammitglieder und der Einsatz von modernen Tools ausschlaggebend.
Speziell die wöchentlichen strukturierten Besprechungen der aktuellen Fortschritte und Probleme der Teammitglieder, aber auch die weitere Planung haben sehr geholfen, den aktuellen Stand des Projektes zu erreichen.
Während des Projektes sind mithilfe der manuellen kontinuierlichen App-Tests bei der Entwicklung neuer Funktionen von den Teammitgliedern regelmäßig Bugs und Verbesserungsvorschläge gefunden, die während den Weekly-Standups dokumentiert und dementsprechend eine neue Aufgabe im Teamboard erstellt wurde.\newline
Trotz anfänglicher technischer Schwierigkeiten beim Einrichten der Entwicklungsumgebungen auf den unterschiedlichen Betriebssystemen, Kompilierfehler durch Änderungen am Sourcecode und Verzögerungen bei der Entwicklung wurden dennoch alle gesetzten Meilensteine erreicht.
Langfristig kann die STH-App vor allem durch die einfache und zuverlässige Architektur und die Benutzerfreundlichkeit ein großes Potenzial in verschiedenen Märkten erreichen.\newline
Während des Projekts konnten wir wertvolle Erfahrungen sammeln, insbesondere in Bezug auf die agile Arbeitsweise und die Nutzung von Tools zur Zusammenarbeit.
Ein entscheidender Erfolgsfaktor war auch das offene Feedback und die kontinuierliche Verbesserung basierend auf den Ergebnissen der App-Tests.

\chapter{Ausblick}

Mit Blick auf die zukünftige Entwicklung der STH-App sind mehrere bedeutende Erweiterungen und Verbesserungen geplant, um die App kontinuierlich zu optimieren und das Benutzererlebnis zu verbessern:

\begin{itemize}
\item Verbesserung des Suchalgorithmus für den Search-Screen: Die Optimierung des Suchalgorithmus wird die Effizienz und Genauigkeit der Suchergebnisse erhöhen, wodurch Nutzer schneller und präziser die gewünschten Informationen finden können.
\item Verbesserung der UI/UX-Implementierung: Ein überarbeitetes und schöneres Design steigert die Benutzerfreundlichkeit der App weiter, was zu einer intuitiven und angenehmeren Benutzererfahrung führt. Dabei kann die Bedienung der App durch gezielte Anpassungen bestehender Funktionen weiter vereinfacht und die Zufriedenheit der Nutzer erhöht werden.
\item Einrichten von In-App-Linkouts für einen direkten Wechsel zu einem anderen Screen: Diese Funktion ermöglicht es Nutzern, direkt zwischen verschiedenen Bildschirmen zu wechseln, was das Navigieren innerhalb der App verbessert.
\item Integration von KI-basierten Features: Durch den Einsatz von künstlicher Intelligenz können personalisierte Empfehlungen und automatisierte Prozesse erzeugt werden, die das Nutzererlebnis noch individueller und effizienter gestalten. Die KI-Features können beispielsweise auf dem Home-Screen oder Search-Screen bei der Suche von Sportlern oder Sportmanagern helfen.
\item Lokalisierung und Übersetzungen von Texten in der App: Die Lokalisierung und Übersetzung der App-Inhalte in verschiedene Sprachen wird die Reichweite der STH-App erweitern und die Benutzerfreundlichkeit für internationale Nutzer erhöhen.
\item Implementierung von Personalisierungsoptionen: Nutzern werden mehr Möglichkeiten zur individuellen Anpassung der App geboten, was deren Engagement und Zufriedenheit fördern wird.
\end{itemize}
Zusätzlich sollen auch technische Versionsupdates des Frameworks Flutter, der dazugehörigen Dependencies sowie von Apple und Google berücksichtigt und der Source-Code der STH-App dementsprechend aktualisiert werden. Dies stellt sicher, dass die App immer auf dem aktuellen Stand der Technik bleibt und optimale Performance bietet.
Weitere wichtige Punkte sind die Performance der STH-App auf den Geräten der Nutzer sowie die Sicherheit und der Datenschutz. 
Mögliche Herausforderungen, wie die Anpassung an verschiedene regionale Anforderungen, werden durch gezielte Forschung und Entwicklung adressiert.
In den kommenden Monaten wird das Projektteam verstärkt auf Innovationsprozesse und Marktanalysen setzen, um die App kontinuierlich zu optimieren und das Benutzererlebnis zu verbessern. Dabei wird eine umfassende Teststrategie entwickelt, die verschiedene Testmethoden wie UI-Tests, A/B-Tests, manuelle Tests, Testautomatisierung, Unit-Tests und Crowdtests umfasst. Dies dient dazu, Benutzerwünsche und -feedback gezielt zu erfassen und ein breites Spektrum an Nutzern zu berücksichtigen.
Langfristig plant das Projektteam die Veröffentlichung der STH-App inklusive Wartung und technischem Support. Eine Community beziehungsweise ein Forum soll aufgebaut werden, um kontinuierliches Feedback zu erhalten und regelmäßige Fehlerbehebungen sowie Verbesserungen über App-Updates umzusetzen.\newline
Abschließend lässt sich feststellen, dass dieses Projekt in der Zukunft zahlreiche Möglichkeiten und Optionen für den Erfolg der STH-App bietet. Durch die geplanten Erweiterungen und die konsequente Ausrichtung auf die Bedürfnisse der Nutzer wird die STH-App ihre Position als innovatives und nützliches Werkzeug für Sportmanager und Sportler weiter festigen und ausbauen.